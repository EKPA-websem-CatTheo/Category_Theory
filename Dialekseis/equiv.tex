\documentclass[11pt]{article}

\usepackage{fullpage,amsthm,amssymb,amsmath,amsfonts}
\usepackage{graphicx,algorithmic,algorithm}
\usepackage[]{datetime}
\usepackage{aliascnt} % new !
\usepackage{hyperref,todonotes}
\usepackage{lmodern}
\usepackage{mathtools}

\usepackage{tikz}
\usepackage{tikz-cd}

% for \mapsfrom
\usepackage{stmaryrd}

\usepackage{subcaption}
\usepackage{cleveref}

\usepackage[greek,english]{babel}
\usepackage[utf8]{inputenc}
\usepackage{alphabeta}

\usepackage{amsmath,ifthen,color,eurosym}
\usepackage{wrapfig,hyperref,cite}
%\usepackage{times}
\usepackage{latexsym,amsthm,amsmath,amssymb,url}
%,cite}

\usepackage{float}

\extrafloats{100}

\usepackage{enumerate}
\usepackage{enumitem}

%\usepackage{dirtytalk}

\usepackage{cite}



% \newcommand{\aliv}[1]{\marginpar{\scriptsize \tt {A.Λ.}> {\sl #1}}}

\usepackage{color}


\newcommand{\yellow}[1]{{\color{Yellow}#1}}
\newcommand{\black}[1]{{\color{Black}#1}}
\newcommand{\other}[1]{{\color{Other}#1}}
\newcommand{\otherother}[1]{{\color{Otherother}#1}}
\newcommand{\blue}[1]{{\color{Blue}#1}}
\newcommand{\red}[1]{{\color{Red}#1}}
\newcommand{\green}[1]{{\color{mygreen}#1}}
\newcommand{\magenta}[1]{{\color{Magenta}#1}}
\newcommand{\brown}[1]{{\color{Brown}#1}}
\newcommand{\white}[1]{{\color{White}#1}}
\newcommand{\grey}[1]{{\color{Grey}#1}}
\DeclareMathOperator{\tw}{{\bf tw}}


% Autoref stuff

\hypersetup{
  colorlinks   = true, %Colours links instead of ugly boxes
  urlcolor     = blue, %Colour for external hyperlinks
  linkcolor    = blue, %Colour of internal links
  citecolor   = red %Colour of citations
}


% for a good theorem numbering
\newtheorem{dummy}{***}[section]
\newcommand{\mynewtheorem}[2]{
 \newaliascnt{#1}{dummy}
 \newtheorem{#1}[#1]{#2}
 \aliascntresetthe{#1}
 % maybe we will squish some autoref defaults, but who cares?
 \expandafter\def\csname #1autorefname\endcsname{#2}
}

\def\problemautorefname{Problem}
\def\chapterautorefname{Chapter}
\def\definitionautorefname{Definition}
\def\claimautorefname{Claim}
\def\corollaryautorefname{Corollary}
\def\lemmaautorefname{Lemma}
\def\sublemmaautorefname{Sublemma}
\def\propositionautorefname{Proposition}
\def\observationautorefname{Observation}
\def\corollaryautorefname{Corollary}
\def\sectionautorefname{Section}
\def\subsectionautorefname{Subsection}
\def\figureautorefname{Diagram}
\def\tableautorefname{Table}
\def\remarkautorefname{Remark}
\def\rmkautorefname{Remark}
\def\observationautorefname{Observation}
\def\conjectureautorefname{Conjecture}
\def\questionautorefname{Question}


\theoremstyle{plain}
\mynewtheorem{theorem}{Theorem}
\mynewtheorem{proposition}{Proposition}
\mynewtheorem{corollary}{Corollary}
\mynewtheorem{lemma}{Lemma}
\mynewtheorem{exercise}{Exercise}

\theoremstyle{definition}
\mynewtheorem{definition}{Definition}
\theoremstyle{remark}
\mynewtheorem{sublemma}{Sublemma}
\mynewtheorem{claim}{Claim}
\mynewtheorem{fact}{Fact}
\mynewtheorem{remark}{Remark}
\mynewtheorem{application}{Application}
\mynewtheorem{conjecture}{Conjecture}
\mynewtheorem{question}{Question}       
\mynewtheorem{observation}{Observation}
\mynewtheorem{problem}{Problem}
\mynewtheorem{note}{Note}
\mynewtheorem{examples}{Examples}
\mynewtheorem{example}{Example}
\mynewtheorem{exercises}{Exercises}

%1-1 correspondence
\makeatletter
\def\DoubleFill@{\arrowfill@=\Relbar=}
\providecommand*\xDouble[2][]{%
  \ext@arrow 0055{\DoubleFill@}{#1}{#2}}
\makeatother

%Subobject classifier
\newcommand{\ceil}[1]{\lceil #1 \rceil }

\newcommand{\cX}{\mathcal{X}}
\newcommand{\cT}{\mathcal{T}}
\newcommand{\cC}{\mathcal{C}}
\newcommand{\cD}{\mathcal{D}}
\newcommand{\cA}{\mathcal{A}}
\newcommand{\cB}{\mathcal{B}}
\newcommand{\cE}{\mathcal{E}}
\newcommand{\cI}{\mathcal{I}}
\newcommand{\cP}{\mathcal{P}}

\newcommand{\op}{\mathrm{op}}
\newcommand{\Hom}{\mathrm{Hom}}
\newcommand{\Ob}{\mathrm{Ob}}
\newcommand{\Ar}{\mathrm{Ar}}
\newcommand{\dom}{\mathrm{dom}}
\newcommand{\cod}{\mathrm{cod}}
\newcommand{\id}{\mathrm{id}}
\newcommand{\Cone}{\mathrm{Cone}}
\newcommand{\Cocone}{\mathrm{Cocone}}
\newcommand{\colim}{\mathrm{colim}}
\newcommand{\Sub}{\mathrm{Sub}}
\newcommand{\PSh}{\mathrm{PSh}}
\newcommand{\im}{\mathrm{im}}
\newcommand{\Cat}{\mathrm{Cat}}
\newcommand{\Alg}{\mathrm{Alg}}
\newcommand{\Adj}{\mathrm{Adj}}
\newcommand{\Cov}{\mathrm{Cov}}
\newcommand{\Sh}{\mathrm{Sh}}
\newcommand{\Sep}{\mathrm{Sep}}
\newcommand{\ClSub}{\mathrm{ClSub}}

\newcommand{\N}{\mathbb{N}}
\newcommand{\R}{\mathbb{R}}



\title{Category Theory and Topos Theory: Lecture Notes}
\author{CatTheo Fan}
\date{\today}

\begin{document}

\maketitle


\begin{figure}[H]
    \centering
    \begin{tikzcd}
  \arrow[rr, bend right=60] \arrow[rr, bend right=49] \arrow[rr, bend right] \arrow[rr, bend left=60] \arrow[rr, bend left=49] \arrow[rr, bend left] \arrow[rr] \arrow[rr, bend right=67] \arrow[rr, bend left=67] \arrow[rr, bend right=71] \arrow[rr, bend left=71] \arrow[rr, bend left=74] \arrow[rr, bend right=74] \arrow[rr, bend left=76] \arrow[rr, bend right=76] \arrow[rr, bend right=78] \arrow[rr, bend left=78] \arrow[rr, bend right=79] \arrow[rr, bend left=79]                                                                                                           &  &  \arrow[dd, bend right=76] \arrow[dd, bend right=74] \arrow[dd, bend right=71] \arrow[dd, bend right=67] \arrow[dd, bend right=60] \arrow[dd, bend right=49] \arrow[dd, bend right] \arrow[dd, bend left=71] \arrow[dd, bend left=74] \arrow[dd, bend left=76] \arrow[dd, bend left=67] \arrow[dd, bend left=60] \arrow[dd, bend left=49] \arrow[dd, bend left] \arrow[dd] \arrow[dd, bend left=78] \arrow[dd, bend right=78] \arrow[dd, bend right=79] \arrow[dd, bend right=80] \\                        &  &                            \\
 \arrow[rruu, bend right=60] \arrow[rruu, bend right=49] \arrow[rruu, bend right] \arrow[rruu, bend left=60] \arrow[rruu, bend left=49] \arrow[rruu, bend left] \arrow[uu, bend left=60] \arrow[uu, bend left=49] \arrow[uu, bend left] \arrow[uu, bend right=60] \arrow[uu, bend right=49] \arrow[uu, bend right] \arrow[uu, bend right=67] \arrow[uu, bend left=67] \arrow[uu, bend left=71] \arrow[uu, bend right=71] \arrow[uu, bend right=74] \arrow[uu, bend left=74] \arrow[uu, bend right=76] \arrow[uu, bend left=76] \arrow[uu, bend right=78] \arrow[uu, bend left=78] \arrow[uu, bend left=79] \arrow[uu, bend right=79] \arrow[uu] \arrow[rruu] \arrow[rr, bend left=79] \arrow[rr, bend left=78] \arrow[rr, bend left=76] \arrow[rr, bend left=74] \arrow[rr, bend left=71] \arrow[rr, bend left=67] \arrow[rr, bend left=60] \arrow[rr, bend left=49] \arrow[rr, bend left] \arrow[rr, bend right=78] \arrow[rr, bend right=76] \arrow[rr, bend right=74] \arrow[rr, bend right=71] \arrow[rr, bend right=67] \arrow[rr, bend right=60] \arrow[rr, bend right=49] \arrow[rr, bend right] \arrow[rr] &  &  \arrow[lluu, bend left=67] \arrow[lluu, bend left=71] \arrow[lluu, bend left=60] \arrow[lluu, bend left=49] \arrow[lluu, bend left] \arrow[lluu, bend left] \arrow[lluu, bend right=71] \arrow[lluu, bend right=67] \arrow[lluu, bend right=49] \arrow[lluu, bend right] \arrow[lluu] \arrow[lluu, bend right=60]                                                                                                                                                               
\end{tikzcd}
\label{psy}
\end{figure}






%Έκατσα και εγραψα για το "κεφι" μου το κομματι με το equivalence. Όποιος γραψει τις υπόλοιπες σημειωσεις της τριτης διαλεξης ας το χρησιμοποιήσει όπως θελει. Αν εχω χρονο θα κανω το ιδιο για το F mono
%Ηθελα και μια μικρή βοηθεια με το τεχ. Αυτα τα δυο διαγραμματακια απο κατω, ιδανικα τα ηθελα διπλα διπλα αλλα και στο κεντρο. Το εψαξα στο google και λενε κατι τρελα με cm. Αν καποιος ειναι εξασκημενος στο tikz και το χει ευκολα να το φτιαξει θα ταν καλο γιατι τωρα μου γαμαει το ocd.


Show that $F:\cC \to \cD$ is an equivalence of categories $\iff$ $F$ is full, faithfull and essentially surjective on objects, meaning $(\forall D \in \cD_0 )(\exists C \in \cC_0) FC\cong D$.\\
Begin with ($\Rightarrow$). Let $F$ be an equivalence of categories. Then we know there exists $G: \cD \to \cC$ and natural transformations $\mu :FG \Rightarrow \id_{\cD}$ and $\nu:GF \Rightarrow \id_{\cC}$.\\
To show $F$ is essentially surjective take an arbitrary $D \in \cD_0$ and write $D=\id_{\cD}(D)\cong FG(D)$. So we can take $C$ to be $G(D) \in \cC_0$.\\
To show $F$ is faithfull take $f,g:A \to B$ with $F(f)=F(g): F(A)\to F(B) \Rightarrow GF(f)=GF(g)$. The naturality squares of $\nu$ for $f,g$ respectively are:

% https://q.uiver.app/?q=WzAsNCxbMCwwLCJHRihBKSJdLFsyLDAsIkdGKEIpIl0sWzIsMiwiQiJdLFswLDIsIkEiXSxbMCwxLCJHRihmKSJdLFsxLDIsIlxcbnVfQiJdLFswLDMsIlxcbnVfQSIsMl0sWzMsMiwiZiIsMl1d
\begin{tikzcd}
	{GF(A)} && {GF(B)} \\
	\\
	{A} && {B}
	\arrow["{GF(f)}", from=1-1, to=1-3]
	\arrow["{\nu_B}", from=1-3, to=3-3]
	\arrow["{\nu_A}"', from=1-1, to=3-1]
	\arrow["{f}"', from=3-1, to=3-3]
\end{tikzcd}
% https://q.uiver.app/?q=WzAsNCxbMCwwLCJHRihBKSJdLFsyLDAsIkdGKEIpIl0sWzIsMiwiQiJdLFswLDIsIkEiXSxbMCwxLCJHRihmKSJdLFsxLDIsIlxcbnVfQiJdLFswLDMsIlxcbnVfQSIsMl0sWzMsMiwiZiIsMl1d
\begin{tikzcd}
	{GF(A)} && {GF(B)} \\
	\\
	{A} && {B}
	\arrow["{GF(g)}", from=1-1, to=1-3]
	\arrow["{\nu_B}", from=1-3, to=3-3]
	\arrow["{\nu_A}"', from=1-1, to=3-1]
	\arrow["{g}"', from=3-1, to=3-3]
\end{tikzcd}\\
with the equalities $f\nu_A=\nu_B GF(f)$ and $g \nu_B=\nu_BGF(g)$. But $\nu_A$ is an iso thus \[f=\nu_B GF(f)\nu_A^{-1}=\nu_B GF(g)\nu_A^{-1}=g\]





To show $F$ is full take $f \in \hom_{\cD}(F(A),F(B))$. We want to find an arrow in $\cC$ thats maps on $f$. The simplest way to "get back" in $\cC$ is to apply the functor $G$. We get \[G(f):GF(A) \to GF(B)\] but since $GF(A) \cong \id_{\cC}(A)=A$ we have an iso $\nu_A:GF(A) \to A$. Thus we can write: 
% https://q.uiver.app/?q=WzAsNCxbMCwwLCJBIl0sWzEsMCwiR0YoQSkiXSxbMiwwLCJHRihCKSJdLFszLDAsIkIiXSxbMCwxLCJcXG51X0Feey0xfSJdLFsxLDIsIkcoZikiXSxbMiwzLCJcXG51X0IiXV0=
\[\begin{tikzcd}
	{A} & {GF(A)} & {GF(B)} & {B}
	\arrow["{\nu_A^{-1}}", from=1-1, to=1-2]
	\arrow["{G(f)}", from=1-2, to=1-3]
	\arrow["{\nu_B}", from=1-3, to=1-4]
\end{tikzcd}\] and take $g=\nu_B G(f) \nu_A^{-1}$. 
%Now consider \[F(g)=F(\nu_B)FG(f)F(\nu_A^{-1})\]
If we take, as above, the naturality square of $\nu$ for $g$ we will obtain $g=\nu_B GF(g)\nu_A^{-1}$.
 Thus, \[GF(g)=G(f)\] With a similar argument to the above we can show $G$ is faithfull and get \[F(g)=f\]
 Now for $(\Leftarrow)$. We wish to show $F$ is an equivalence of categories. This amounts to defining a functor $G:\cD \to \cC$ for whom we have $GF\cong \id_{\cC}$ and $FG \cong \id_{\cD}$.\\
  To define $G$ on $\cD_0$ we use the fact that $F$ is essentially surjective thus $\forall D \exists C: F(C) \cong D$. This is the only procedure, at hand, that associates objects in $\cD$ to objects in $\cC$. We "have no other choice than" to define $G(D)=C$.\\
  When it comes to arrows we take an $f':A' \to B'$. Then there exist $A,B \in \cC_0$ so that $F(A)\cong A'$ and $F(B)\cong B'$ thus we get the commuutative diagram: 
  % https://q.uiver.app/?q=WzAsNCxbMCwwLCJBJyJdLFsyLDAsIkInIl0sWzAsMiwiRihBKSJdLFsyLDIsIkYoQikiXSxbMCwxLCJmIl0sWzAsMiwiIiwyLHsib2Zmc2V0IjotMn1dLFsyLDMsIiIsMix7InN0eWxlIjp7ImJvZHkiOnsibmFtZSI6ImRhc2hlZCJ9fX1dLFsyLDAsIlxcdGF1XzEiLDAseyJvZmZzZXQiOi0yfV0sWzMsMSwiIiwyLHsib2Zmc2V0IjotMn1dLFsxLDMsIlxcdGF1XzIiLDAseyJvZmZzZXQiOi0yfV1d
\[\begin{tikzcd}
	{A'} && {B'} \\
	\\
	{F(A)} && {F(B)}
	\arrow["{f'}", from=1-1, to=1-3]
	\arrow[from=1-1, to=3-1, shift left=2]
	\arrow[from=3-1, to=3-3, dashed]
	\arrow["{\tau_1}", from=3-1, to=1-1, shift left=2]
	\arrow[from=3-3, to=1-3, shift left=2]
	\arrow["{\tau_2}", from=1-3, to=3-3, shift left=2]
\end{tikzcd}\]
 So $f'$ corresponds to an arrow $\tau_2 f' \tau_1^{-1}=:f:F(A)\to F(B)$. But since $F$ is full and faithfull there exists a bijection between the Hom-sets $\hom_{\cD}(F(A),F(B))$ and $\hom_{\cC}(A,B)$. Let $g$ be the unique arrow that is the inverse image of $f$ under this mapping. We define $G(f)=g$.\\
 Firstly we must show $G$ is a functor. Take $\id_D \in \cD_1$, and as above an object $C$ such that $F(C) \cong D$ by a morphism,say $\tau$. Notice this also means $G(D)=C$. According to the above procedure, we take the inverse image of $\tau^{-1} \id_D \tau=\id_{F(C)}$. Evidently $F(\id_C)=\id_{F(C)}$ and since the aforementioned correspondence is a bijection, $\id_C$ is the unique such arrow. So indeed, $G(\id_D)=\id_C=\id_{G(D)}$\\
 Take $gf:A' \to B' \to E'$ as before we can find objects$A,B,E$ such that e.g. $\tau_3:F(E) \cong E'$. 
 % https://q.uiver.app/?q=WzAsNixbMCwwLCJBJyJdLFsyLDAsIkInIl0sWzAsMiwiRihBKSJdLFsyLDIsIkYoQikiXSxbNCwwLCJFJyJdLFs0LDIsIkYoRSkiXSxbMCwxLCJmIl0sWzAsMiwiIiwyLHsib2Zmc2V0IjotMn1dLFsyLDMsIiIsMix7InN0eWxlIjp7ImJvZHkiOnsibmFtZSI6ImRhc2hlZCJ9fX1dLFsyLDAsIlxcdGF1XzEiLDEseyJvZmZzZXQiOi0yfV0sWzMsMSwiXFx0YXVfMiIsMSx7Im9mZnNldCI6LTJ9XSxbMSwzLCIiLDAseyJvZmZzZXQiOi0yfV0sWzEsNCwiZyJdLFszLDUsIiIsMCx7InN0eWxlIjp7ImJvZHkiOnsibmFtZSI6ImRhc2hlZCJ9fX1dLFs1LDQsIlxcdGF1XzMiLDEseyJvZmZzZXQiOi0yfV0sWzQsNSwiIiwwLHsib2Zmc2V0IjotMn1dXQ==
\[\begin{tikzcd}
	{A'} && {B'} && {E'} \\
	\\
	{F(A)} && {F(B)} && {F(E)}
	\arrow["{f}", from=1-1, to=1-3]
	\arrow[from=1-1, to=3-1, shift left=2]
	\arrow[from=3-1, to=3-3, dashed]
	\arrow["{\tau_1}" description, from=3-1, to=1-1, shift left=2]
	\arrow["{\tau_2}" description, from=3-3, to=1-3, shift left=2]
	\arrow[from=1-3, to=3-3, shift left=2]
	\arrow["{g}", from=1-3, to=1-5]
	\arrow[from=3-3, to=3-5, dashed]
	\arrow["{\tau_3}" description, from=3-5, to=1-5, shift left=2]
	\arrow[from=1-5, to=3-5, shift left=2]
\end{tikzcd}\]
We have $h=G(gf)$ to be the inverse image of $\tau_3^{-1} gf \tau_1$.One the other hand, $G(g)G(f)$ is the composition of the inverse images of: \[ (\tau_3^{-1} g \tau_2) (\tau_2^{-1} f \tau_1)=\tau_3^{-1} gf \tau_1\] 
 Thus indeed $G(gf)=G(g)G(f)$.\\
 We want to define a natural transformation with a certain property. In our hypotheses we notice that $F$ being essentially surjective gives for any object $D \in \cD_0$ a morphism $\tau_D \in \cD_1$, which is also an iso. Thus we can readily define a naural transformation $\tau=\{\tau_D|D \in \cD_0\}$ by "putting these morphisms together".
 It turns out it is easier to show $\tau^{-1}:GF\cong \id_{\cC}$. Recall that $G(D)=C$ where $F(C)\cong D \Rightarrow GF(C) \cong G(D)=C$. If we take an arrow $f':A' \to B'$ in $\cD_1$ as above $g=G(f)$ is the inverse image of $f$ under the mapping $\hom_{\cC}(A,B) \to \hom_{\cD}(F(A),F(B))$, so if we apply $F$ we will get $f$ again, but $F(g)=f=\tau_2 f' \tau_1^{-1} \Rightarrow \tau_2^{-1}f=f'\tau^{-1}$ which is preciecly the naturality square of $\tau^{-1}$ for $f$.
  This suffices by ex. 23. The family of inverses will be $(\tau^{-1})^{-1}=\tau:GF \Rightarrow \id_{\cC}$.
  %It becomes clear that the natural transformation which gives $GF\cong \id_{\cC}$ is given by the family of morphisms we get from $F$ being essentially surjective.
  \end{document}