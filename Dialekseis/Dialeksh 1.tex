\documentclass[11pt]{article}

\usepackage{fullpage,amsthm,amssymb,amsmath,amsfonts}
\usepackage{graphicx,algorithmic,algorithm}
\usepackage[]{datetime}
\usepackage{aliascnt} % new !
\usepackage{hyperref,todonotes}
\usepackage{lmodern}
\usepackage{mathtools}

\usepackage{tikz}
\usepackage{tikz-cd}

% for \mapsfrom
\usepackage{stmaryrd}

\usepackage{subcaption}
\usepackage{cleveref}

\usepackage[greek,english]{babel}
\usepackage[utf8]{inputenc}
\usepackage{alphabeta}

\usepackage{amsmath,ifthen,color,eurosym}
\usepackage{wrapfig,hyperref,cite}
%\usepackage{times}
\usepackage{latexsym,amsthm,amsmath,amssymb,url}
%,cite}

\usepackage{float}

\extrafloats{100}

\usepackage{enumerate}
\usepackage{enumitem}

%\usepackage{dirtytalk}

\usepackage{cite}
\usepackage{enumitem}
\setlist[enumerate]{resume}
\usepackage{amsthm}


% \newcommand{\aliv}[1]{\marginpar{\scriptsize \tt {A.Λ.}> {\sl #1}}}

\usepackage{color}


\newcommand{\yellow}[1]{{\color{Yellow}#1}}
\newcommand{\black}[1]{{\color{Black}#1}}
\newcommand{\other}[1]{{\color{Other}#1}}
\newcommand{\otherother}[1]{{\color{Otherother}#1}}
\newcommand{\blue}[1]{{\color{Blue}#1}}
\newcommand{\red}[1]{{\color{Red}#1}}
\newcommand{\green}[1]{{\color{mygreen}#1}}
\newcommand{\magenta}[1]{{\color{Magenta}#1}}
\newcommand{\brown}[1]{{\color{Brown}#1}}
\newcommand{\white}[1]{{\color{White}#1}}
\newcommand{\grey}[1]{{\color{Grey}#1}}
\DeclareMathOperator{\tw}{{\bf tw}}


% Autoref stuff

\hypersetup{
  colorlinks   = true, %Colours links instead of ugly boxes
  urlcolor     = blue, %Colour for external hyperlinks
  linkcolor    = blue, %Colour of internal links
  citecolor   = red %Colour of citations
}




\newtheoremstyle{plain}% <name>
{3pt} % <space above>
{3pt} % <space below>
{\it} % <Body font>
{} % <Indent amount>
{\rm\bfseries} % <Theorem head font>
{.} % <Punctuation after theorem head>
{.5em} % <Space after theorem head>
{} % <Theorem head spec (can be left empty, meaning ‘normal’)>

\newtheoremstyle{definition}% <name>
{15pt} % <space above>
{15pt} % <space below>
{\rm} % <Body font>
{} % <Indent amount>
{\rm\bfseries} % <Theorem head font>
{.} % <Punctuation after theorem head>
{.5em} % <Space after theorem head>
{} % <Theorem head spec (can be left empty, meaning ‘normal’)>

\newtheoremstyle{remark}% <name>
{3pt} % <space above>
{3pt} % <space below>
{\rm} % <Body font>
{} % <Indent amount>
{\it} % <Theorem head font>
{.} % <Punctuation after theorem head>
{.5em} % <Space after theorem head>
{} % <Theorem head spec (can be left empty, meaning ‘normal’)>



%for a good theorem numbering
\newtheorem{dummy}{***}[section]
\newcommand{\mynewtheorem}[2]{
\newaliascnt{#1}{dummy}
\newtheorem{#1}[#1]{#2}
\aliascntresetthe{#1}
 % maybe we will squish some autoref defaults, but who cares?
\expandafter\def\csname #1autorefname\endcsname{#2}
}


\theoremstyle{plain}
\mynewtheorem{theorem}{Theorem}
\mynewtheorem{proposition}{Proposition}
\mynewtheorem{corollary}{Corollary}
\mynewtheorem{lemma}{Lemma}

\theoremstyle{definition}
\mynewtheorem{definition}{Definition}
\mynewtheorem{example}{Example}
\mynewtheorem{exercise}{Exercise}
\mynewtheorem{examples}{Examples}
\mynewtheorem{exercises}{Exercises}

\theoremstyle{remark}
\mynewtheorem{claim}{Claim}
\mynewtheorem{remark}{Remark}
\mynewtheorem{conjecture}{Conjecture}
\mynewtheorem{question}{Question}       
\mynewtheorem{observation}{Observation}
\mynewtheorem{problem}{Problem}
\mynewtheorem{note}{Note}
\mynewtheorem{remarks}{Remarks}



\def\problemautorefname{Problem}
\def\chapterautorefname{Chapter}
\def\definitionautorefname{Definition}
\def\claimautorefname{Claim}
\def\corollaryautorefname{Corollary}
\def\lemmaautorefname{Lemma}
\def\sublemmaautorefname{Sublemma}
\def\propositionautorefname{Proposition}
\def\observationautorefname{Observation}
\def\corollaryautorefname{Corollary}
\def\sectionautorefname{Section}
\def\subsectionautorefname{Subsection}
\def\figureautorefname{Figure}
\def\tableautorefname{Table}
\def\remarkautorefname{Remark}
\def\rmkautorefname{Remark}
\def\observationautorefname{Observation}
\def\conjectureautorefname{Conjecture}
\def\questionautorefname{Question}
\def\exampleautorefname{Example}




%1-1 correspondence
\makeatletter
\def\DoubleFill@{\arrowfill@=\Relbar=}
\providecommand*\xDouble[2][]{%
  \ext@arrow 0055{\DoubleFill@}{#1}{#2}}
\makeatother

%Subobject classifier
\newcommand{\ceil}[1]{\lceil #1 \rceil }

\newcommand{\cX}{\mathcal{X}}
\newcommand{\cT}{\mathcal{T}}
\newcommand{\cC}{\mathcal{C}}
\newcommand{\cD}{\mathcal{D}}
\newcommand{\cA}{\mathcal{A}}
\newcommand{\cB}{\mathcal{B}}
\newcommand{\cE}{\mathcal{E}}
\newcommand{\cI}{\mathcal{I}}
\newcommand{\cP}{\mathcal{P}}
\newcommand{\cF}{\mathcal{F}}

\newcommand{\op}{\mathrm{op}}
\newcommand{\Hom}{\mathrm{Hom}}
\newcommand{\Ob}{\mathrm{Ob}}
\newcommand{\Ar}{\mathrm{Ar}}
\newcommand{\dom}{\mathrm{dom}}
\newcommand{\cod}{\mathrm{cod}}
\newcommand{\id}{\mathrm{id}}
\newcommand{\Cone}{\mathrm{Cone}}
\newcommand{\Cocone}{\mathrm{Cocone}}
\newcommand{\colim}{\mathrm{colim}}
\newcommand{\Sub}{\mathrm{Sub}}
\newcommand{\PSh}{\mathrm{PSh}}
\newcommand{\im}{\mathrm{im}}
\newcommand{\Cat}{\mathrm{Cat}}
\newcommand{\Alg}{\mathrm{Alg}}
\newcommand{\Adj}{\mathrm{Adj}}
\newcommand{\Cov}{\mathrm{Cov}}
\newcommand{\Sh}{\mathrm{Sh}}
\newcommand{\Sep}{\mathrm{Sep}}
\newcommand{\ClSub}{\mathrm{ClSub}}

\newcommand{\N}{\mathbb{N}}
\newcommand{\R}{\mathbb{R}}

\title{Category Theory: Lecture 1}
\author{$\mathrm{CT_{\Epsilon\Kappa\Pi\Alpha}}$\MakeLowercase{WS}}
\date{\today}

\begin{document}

\maketitle

\begin{figure}[H]
    \centering
    \begin{tikzcd}
  \arrow[rr, bend right=60] \arrow[rr, bend right=49] \arrow[rr, bend right] \arrow[rr, bend left=60] \arrow[rr, bend left=49] \arrow[rr, bend left] \arrow[rr] \arrow[rr, bend right=67] \arrow[rr, bend left=67] \arrow[rr, bend right=71] \arrow[rr, bend left=71] \arrow[rr, bend left=74] \arrow[rr, bend right=74] \arrow[rr, bend left=76] \arrow[rr, bend right=76] \arrow[rr, bend right=78] \arrow[rr, bend left=78] \arrow[rr, bend right=79] \arrow[rr, bend left=79]                                                                                                           &  &  \arrow[dd, bend right=76] \arrow[dd, bend right=74] \arrow[dd, bend right=71] \arrow[dd, bend right=67] \arrow[dd, bend right=60] \arrow[dd, bend right=49] \arrow[dd, bend right] \arrow[dd, bend left=71] \arrow[dd, bend left=74] \arrow[dd, bend left=76] \arrow[dd, bend left=67] \arrow[dd, bend left=60] \arrow[dd, bend left=49] \arrow[dd, bend left] \arrow[dd] \arrow[dd, bend left=78] \arrow[dd, bend right=78] \arrow[dd, bend right=79] \arrow[dd, bend right=80] \\                        &  &                            \\
 \arrow[rruu, bend right=60] \arrow[rruu, bend right=49] \arrow[rruu, bend right] \arrow[rruu, bend left=60] \arrow[rruu, bend left=49] \arrow[rruu, bend left] \arrow[uu, bend left=60] \arrow[uu, bend left=49] \arrow[uu, bend left] \arrow[uu, bend right=60] \arrow[uu, bend right=49] \arrow[uu, bend right] \arrow[uu, bend right=67] \arrow[uu, bend left=67] \arrow[uu, bend left=71] \arrow[uu, bend right=71] \arrow[uu, bend right=74] \arrow[uu, bend left=74] \arrow[uu, bend right=76] \arrow[uu, bend left=76] \arrow[uu, bend right=78] \arrow[uu, bend left=78] \arrow[uu, bend left=79] \arrow[uu, bend right=79] \arrow[uu] \arrow[rruu] \arrow[rr, bend left=79] \arrow[rr, bend left=78] \arrow[rr, bend left=76] \arrow[rr, bend left=74] \arrow[rr, bend left=71] \arrow[rr, bend left=67] \arrow[rr, bend left=60] \arrow[rr, bend left=49] \arrow[rr, bend left] \arrow[rr, bend right=78] \arrow[rr, bend right=76] \arrow[rr, bend right=74] \arrow[rr, bend right=71] \arrow[rr, bend right=67] \arrow[rr, bend right=60] \arrow[rr, bend right=49] \arrow[rr, bend right] \arrow[rr] &  &  \arrow[lluu, bend left=67] \arrow[lluu, bend left=71] \arrow[lluu, bend left=60] \arrow[lluu, bend left=49] \arrow[lluu, bend left] \arrow[lluu, bend left] \arrow[lluu, bend right=71] \arrow[lluu, bend right=67] \arrow[lluu, bend right=49] \arrow[lluu, bend right] \arrow[lluu] \arrow[lluu, bend right=60]      
\end{tikzcd} 
\end{figure}


\section{Definition of a Category and Examples.}

\begin{definition}[Category]
A {\em category} $\cC$ is given by a collection $\cC_{0}$ of 
{\em objects} and a collection $\cC_{1}$ of {\em arrows} or
{\em morphisms} which have the following structure:
\begin{itemize}
    \item Each arrow has a {\em domain} and a {\em codomain}
    which are objects. We write $f:X\to Y$ or $X\xrightarrow{f}Y$
    if $X$ is the domain of $f$ and $Y$ its codomain.
    We also write $X=\dom(f)$ and $Y=\cod(f)$.
    
    \item Given two arrows $f$ and $g$ such that
    $\cod(f)=\dom(g)$, the {\em composition} of $f$ and $g$ is
    defined and has domain $\dom(f)$ and codomain $\cod(g)$. We
    write $g\circ f$ or $gf$:
    \begin{equation*}
        (X\xrightarrow{f}Y\xrightarrow{g}Z)\mapsto
        (X\xrightarrow{g\circ f}Z).
    \end{equation*}
    
    \item Composition is {\em associative}, that is: given morphisms
    \begin{equation*}
        X\xrightarrow{f}Y\xrightarrow{g}Z\xrightarrow{h}W
    \end{equation*}
    we get $h\circ(g\circ f)=(h\circ g)\circ f$.
    
    \item For every object $X$ there is an {\em identity} arrow
    $\id_{X}:X\to X$, such that for every $g:Y\to X$ and
    $f:X\to Y$ we get $\id_{X}\circ g=g$ and $f\circ\id_{X}=f$.
\end{itemize}
\end{definition}

\begin{remark}
For every object $X\in\cC_{0}$ the identity arrow $\id_{X}$ is unique
(see \autoref{exc:identity}).
\end{remark}

\begin{definition}[$\Hom$]
Given two objects $X$ and $Y$, the collection of all arrows $f$ in 
$\cC$ with $\dom(f)=X$ and $\cod(f)=Y$ is denoted by $\Hom(X,Y)$.
\end{definition}

\begin{examples}
\begin{enumerate}
    \item $\mathbf{1}$ is the category with one object,
    $*$, and one arrow, $id_{*}$.
    \item $\mathbf{0}$ is the empty category. It has no objects
    and no arrows.
    \item Most basic categories have as objects certain mathematical
    structures, and the structure-preserving functions as morphisms. Examples:
    \begin{enumerate}
        \item \textbf{Grp} is the category of groups and group homomorphisms.
        \item \textbf{Rng} is the category of rings and ring homomorphisms.
        \item \textbf{Top} is the category of topological spaces and continuous
        functions.
        \item \textbf{Grph} is the category of graphs and graphs homomorphisms.
        \item \textbf{Pos} is the category of partially ordered sets and monotone
        functions.
         \item \textbf{Set} is the category which has the class of all sets as
         objects,and functions between sets as arrows.
    \end{enumerate}
    
    \item Given a set $A$ we can define the category $\cC$ with objects
    the elements of the set $A$ and arrows the identity arrows.
    We call this category the {\em discrete category} on $A$.
    
    \item \label{example:preorder}
    A {\em preorder} is a set $X$ together with a binary
    relation $\leq$ which is reflexive (i.e. $x\leq x,$ for all $x\in
    X$)
    and transitive (i.e. $x\leq y$ and $y \leq z$ imply that $x\leq z$
    for every $x,y,z \in X).$ This can be viewed as a category with
    objects the elements of $X$ and for every two elements (objects)
    $x,y\in A$ exactly one arrow $x\to y$.
    The "converse" is also true (See \autoref{exc:preorder}).
   
    \item Let ($G$, $\cdot$) be a group. This can be considered as
    a category with one object, denoted by $G$, and with arrows the
    elements of $G$ considered as arrows $G\to G$.
   
\end{enumerate}
\end{examples}

\begin{definition}[Isomorphism]
A morphism $f:X\rightarrow Y$ is an isomorphism if there is $g:Y \rightarrow X$
such that $f\circ g= id_{Y}$ and $g \circ f= id_{X}$. We call $g$ the inverse of
$f$ (and vice versa, of course); it is unique if it exists. We also write
$g=f^{-1}$.
\end{definition}

\begin{definition}[Small - Locally Small Category]
A category $\cC$ is called {\em small} if the collection of arrows of $\cC$ is
a set. A category $\cC$ is called {\em locally small} if for every two objects
$X,Y$ it holds that $\Hom(X,Y)$ is a set.
\end{definition}

\begin{remarks}
\begin{enumerate}
    \item Obviously a small category is also locally small, although the 
    converse does not hold necessarily
    (see \autoref{example:setsmallbutnotlocally} below).
    
    \item Observe that if we have a small category then the collection of its
    objects is also a set. This follows from the fact that there is an
    one-to-one correspondence between the objects and the collection of identity
    arrows, which is a subset of $\cC_{1}$.\footnote{More accurately, this
    follows from the Axiom of Replacement.}
\end{enumerate}
\end{remarks}

\begin{example}\label{example:setsmallbutnotlocally}
Consider the category $\textbf{Set}$. This is a locally small category as
for every pair of sets $X,Y$, $\Hom(X,Y)$ is a set.
However, it is not a small category as the collection of all sets
is not a set.
\end{example}
\begin{examples}
\begin{enumerate}
    \item A {\em groupoid} is a small category in which all the arrows
    are isomorphisms. Observe that a groupoid with one object defines a
    group.
    
    \item Α {\em monoid} is a set $X$ together with a binary operation,
    written like multiplication: $xy$ for $x,y \in X$, which is associative
    and has
    a unit element $e \in X$, satisfying $ex=xe=x$ for all $x \in X$. 
    A monoid can be viewed as a category with one object $*$ and an arrow
    $x:*\to *$ for every $x \in X$.
\end{enumerate}
\end{examples}





\section{Definition of a Functor and Examples.}

\begin{definition}[Functor]
Let $\cC$ and $\cD$ be two categories. A {\em functor} 
$F:\cC\to\cD$ consists of operations $F_{0}:\cC_{0}\to\cD_{0}$
and $F_{1}:\cC_{1}\to\cD_{1}$, such that for each morphism
$f:X\to Y$,
\begin{equation*}
    F_{1}(f):F_{0}(X)\to F_{0}(Y)
\end{equation*}
and:
\begin{itemize}
    \item for two morphisms
    \begin{equation*}
        X\xrightarrow{f}Y\xrightarrow{g}Z
    \end{equation*}
    we have $F_{1}(gf)=F_{1}(g)F_{1}(f)$.
    \item for each $X\in\cC_{0}$,
    $F_{1}(\id_{X})=\id_{F_{0}(X)}$.
\end{itemize}
\end{definition}


\begin{note}\label{note:associativityfunc}
The associativity for functors holds as well (see \autoref{exc:assoc.func}).
\end{note}



\begin{examples}\label{examples:functors}
\begin{enumerate}
   \item For every category $\cC$ there is the {\em identity functor} which
   corresponds every object $X$ to $X$ and every arrow $f$ to $f$.
   
   \item We can now define the category of all categories, denoted as
   $\textbf{Cat}$. This has as objects the categories and as
   arrows the functors between categories.
   The fact that this is a category follows immediately 
   from the associativity of functors.
   
   \item For every category $\cC$ there is the unique functor $\cC
   \rightarrow \mathbf{1}$ and a unique one $\textbf{0} \rightarrow \cC$.
   
   \item There is a functor $U: \textbf{Grp} \rightarrow \textbf{Set}$ which
   assigns to any group $G$ its underlying set. We call this functor
   "forgetful": it "forgets" the mathematical structure. Similarly, there
   are
   forgetful functors $\textbf{Top} \rightarrow \textbf{Set}$,
   $\textbf{Grph}
   \rightarrow \textbf{Set} , \textbf{Rng} \rightarrow \textbf{Set},
    \textbf{Pos} \rightarrow \textbf{Set}$ etcetera.
    Note that we can define forgetful functors that their codomain is not 
    $\textbf{Set.}$
    For example, there is a forgetful functor
    $\textbf{Rng}\rightarrow
    \textbf{AbGrp}$, where $\textbf{AbGrp}$ is the category of Abelian
    Groups. In this case the functor forgets the multiplicative operations
    of
    the ring and therefore an abelian group is defined. Observe that the
    diagram in \autoref{fig:forget} is commutative.
    \begin{figure}[h]
    \centering
    \begin{tikzcd}[column sep=tiny, row sep=scriptsize]
	{} & {} & {} &&&& {} \\
	{} & {} & {(R,+,\cdot)} & { \textbf{Rng}} & {} & { \textbf{Set}} & {R }
	\\
	&&& {} & {\textbf{AbGrp} } \\
	&&& {} & {(R,+)} &&&& {} \\
	&&& {} \\
	&&& {}
	\arrow[from=2-4, to=2-6]
	\arrow[from=2-4, to=3-5]
	\arrow[from=3-5, to=2-6]
	\arrow[from=2-3, to=4-5, maps to]
	\arrow[from=2-3, to=2-7, maps to, bend left]
	\arrow[from=4-5, to=2-7, maps to]
    \end{tikzcd}
    \caption{Composition of forgetful functors.}
    \label{fig:forget}
    \end{figure}

    \item Given two categories $\cC$ and $\cD$ we can define the {\em
    product
    category} $\cC \times \cD$ which has as objects pairs $(C,D) \in
    \cC_{0}
    \times \cD_{0}$, and as arrows: $(C,D) \rightarrow (C',D')$ pairs
    $(f,g)$
    with $f:C \rightarrow C'$ in $\cC$ and $g:D \rightarrow D'$ in $\cD$.
    There are {\em projection functors} 
    $\pi_{0}:\cC \times \cD \rightarrow \cC$ and 
    $\pi_{1}:\cC \times \cD \rightarrow \cD$. (See \autoref{exc:prodcat})
   \end{enumerate}
   
\end{examples}

\begin{definition}[Slice category] \label{def:slicecat}
    Let $\cC$ be a category and $X$ an object of $\cC$. The {\em slice
    category} $\cC/X$ has as objects all arrows $g$ which have codomain
    $X$. An arrow from $g:Y \rightarrow X$ to $h:Z \rightarrow X$ in
    $\cC/X$ 
    is an arrow $k:Y \rightarrow Z$ in $\cC$ which makes the diagram: 
    \begin{figure}[H]
       \centering
       \begin{tikzcd}[column sep=small]
          Y \arrow[rr, "k"] \arrow[rd, "g"'] &   & Z \arrow[ld, "h"] \\
                                   & X &                  
       \end{tikzcd}
         \label{fig:my_label}
    \end{figure}
    commute\footnote{ We say that this diagram {\em commutes} if we mean that
    $hk=g$.}.
\end{definition}

\begin{note}\label{slice:func}
Let $\cC$ be a category. Any morphism $f:A \rightarrow B$ in $\cC$ induces a
functor with domain the $\cC/A$ and codomain $\cC/B$ as follows:
\begin{figure}[h]
    \centering
\begin{tikzcd}[row sep=small,column sep=tiny]
F:                                  & \mathcal{C}/A     &                    & {} \arrow[rrrr]                  &  &  &    & {}                                          & \mathcal{C}/B    &                            \\
                                    & (X \arrow[r, "g"] & A)                 & {} \arrow[rrr, "F_{0}", maps to] &  &  & {} & (X \arrow[r, "g"]                           & A \arrow[r, "f"] & B)                         \\
X \arrow[rr, "k"] \arrow[rdd, "g"'] &                   & Y \arrow[ldd, "h"] & {} \arrow[rrr, "F_1", maps to]   &  &  & {} & X \arrow[rr, "k"] \arrow[rdd, "f \circ g"'] &                  & Y \arrow[ldd, "f \circ h"] \\
                                    &                   &                    &                                  &  &  &    &                                             &                  &                            \\
                                    & A                 &                    &                                  &  &  &    &                                             & B                &                           
\end{tikzcd}
\end{figure}
See \autoref{exc:slicecat} for verification and the fact that this operation
is {\em functorial}, i.e. defines a functor.
\end{note}

\begin{example}
Take the category $\textbf{Set}$ and a singleton $\{*\}$. Then, the category
$\textbf{Set}/\{*\}$ has as objects all the arrows $A \rightarrow \{*\}$.
As far as for the morphisms of $\textbf{Set}/\{*\}$ we have: for every pair
of arrows $f:A\rightarrow \{*\},g:B \rightarrow \{*\}$ we take all the
arrows $k:A \rightarrow B$ that make the diagram below commutative:
\begin{figure}[H]
\centering
\begin{tikzcd}[column sep=small]
    A \arrow[rr, "k"] \arrow[rd, "g"'] &   & B \arrow[ld, "h"] \\
    & \{*\} &                  
\end{tikzcd}
\end{figure}

Note that for each set $A$ there is a unique map $A\to \{*\}$, the one that
sends every element of the set $A$ to $*$. Hence, the above diagram always 
commutes. Therefore, when we speak of the category $\textbf{Set}/\{*\}$,
we practically speak of the category $\textbf{Set}$.
\end{example}

\begin{definition}[Opposite category]
    Given a category $\cC$ we can form a category $\cC^{\op}$ which has the
    same objects and arrows as $\cC$, but with reversed direction; so if
    $f:X\rightarrow Y$ in $\cC$ then $f:Y \rightarrow X$ in $\cC^{\op}$. To
    make it notationally clear, write $\bar{f}$ for the arrow $Y \rightarrow
    X$ corresponding to $f: X \rightarrow Y$ in $\cC$. Composition in
    $\cC^{\op}$ is defined by: if $X\xrightarrow{f} Y\xrightarrow{g} Z$ are
    arrows in $\cC$ then
    \begin{equation*}
        \overline{f}\overline{g}:=\overline{gf}
    \end{equation*}
\end{definition}

\begin{note}\label{note:opposite}
\begin{enumerate}

\item[(a)] In our terminology, a {\em contravariant functor} $F$ from a category
$\cC$ to a category $\cD$ is just a functor from $\cC^{\op}$ to $\cD$, so
it basically inverts the direction of the arrows. A {\em covariant 
functor} is just a functor.
    
\item [(b)] Any functor $F: \cC \rightarrow \cD$ gives rise to a functor
$F^{\op}: \cC^{\op} \rightarrow \cD^{\op}.$
In fact, we have a functor $(-)^{\op}: \Cat \rightarrow \Cat$
(See \autoref{exc:oppcat}).
\end{enumerate}
\end{note}


\begin{examples}\label{example:yoneda}
\begin{enumerate}
Let $\cC$ be a locally small category, $A$ an object in
$\cC$ and $f:X \rightarrow Y$ an arrow in $\cC$.

\item We can define the functor $y_{A}=\Hom(-,A)$ as follows:
\begin{figure}[H]
\centering
\begin{tikzcd}[column sep=tiny,row sep=tiny]
	&&& {} \\
	& {y_A:} & {\mathcal{C}^{\op}} & {} & {\mathbf{Set}} & {} & {} \\
	& {} & {X} && {\text{Hom}(X,A)} & {} \\
	{} & {(Y} & {X)} & {} & {\text{Hom}(Y,A)} & {\text{Hom}(X,A)} & {} \\
	&&&& {(Y \overset{g}\longrightarrow A)} &
	{(X\overset{f}\longrightarrow Y \overset{g}\longrightarrow A)} & {}
	\arrow[from=2-3, to=2-5]
	\arrow[from=3-3, to=3-5, maps to]
	\arrow[from=5-5, to=5-6, maps to]
	\arrow["{\overline{f}}", from=4-2, to=4-3]
	\arrow[from=4-3, to=4-5, maps to]
	\arrow[from=4-5, to=4-6]
\end{tikzcd}
\end{figure} 
  
The functor $y_{A}$ is often referred as {\em Yoneda functor} or
{\em Hom functor}.

\item We also define the $\Hom(A,-)$ functor:
\begin{figure}[H]
  \centering
  \begin{tikzcd}[column sep=tiny,row sep=tiny]
	&&& {} \\
	& {h_A:} & {\mathcal{C}} & {} & {\mathbf{Set}} & {} & {} \\
	& {} & {X} && {\text{Hom}(A,X)} & {} \\
	{} & {(X} & {Y)} & {} & {\text{Hom}(A,X)} & {\text{Hom}(A,Y)} & {} \\
	&&&& {(A \overset{g}\longrightarrow X)} & {(A\overset{g}\longrightarrow
	X\overset{f}\longrightarrow Y)} & {}
	\arrow[from=2-3, to=2-5]
	\arrow[from=3-3, to=3-5, maps to]
	\arrow[from=5-5, to=5-6, maps to]
	\arrow["{f}", from=4-2, to=4-3]
	\arrow[from=4-3, to=4-5, maps to]
	\arrow[from=4-5, to=4-6]
\end{tikzcd}
      \end{figure}
\end{enumerate}

See \autoref{exc:yonedafunc} for verification.
\end{examples}


\section{Exercises}


\begin{exercise}\label{exc:identity}
 For every object $C$ in a category $\cC$ the identity arrow $\id_{C}:C\to C$
is unique, in the sense that there is a unique arrow $C\to C$ with the 
property of the identity.
\end{exercise}

\begin{exercise}\label{exc:preorder}
Prove that the category defined in Example~1.4(\ref{example:preorder}) from a preorder
is indeed a
category. Conversely, if $\cC$ is a small category in which for every two objects
$X,Y$ there is at most one arrow $X\to Y$, defines a preorder, in the sense that $\cC_{0}$
can be turned into a preordered set.
\end{exercise}

\begin{exercise}\label{exc:assoc.func}
Prove that if $F:\cC\to\cD, G:\cD\to\cE, H:\cE\to\cF$ are functors then it holds that
$H\circ(G\circ F)=(H\circ G)\circ F$.
\end{exercise}

\begin{exercise}[Free functor]\label{exc:freefunctor}
Let $A$ be a set and define a set $A^{-1}$ to be an arbitrary set 
disjoint to $A$ and in 1-1 correspondence with $A$, where for each element
$a\in A$ we denote by $a^{-1}$ its corresponding element. Consider now the set
$\Tilde{A}$ of strings on the alphabet $A\cup A^{-1}$, i.e. finite sequences of
elements of $A\cup A^{-1}$.

We define a relation $\sim$ on $\Tilde{A}$ as follows: if $\vec{a}=a_{1}\cdots a_{n},
\vec{b}=b_{1}\cdots b_{m}\in\Tilde{A}$ then $\vec{a}\sim\vec{b}$ iff
we end up with the same string after deleting every substring of the
for $xx^{-1}, x\in A,$ from each $\vec{a},\vec{b}$.
Prove that $\sim$ defines an equivalence relation on $\Tilde{A}$.

Consider the quotient $A/\sim$ of the set $A$ via the relation $\sim$ and
define an operation on $\Tilde{A}/\sim$: let
$[\vec{a}],[\vec{b}]\in\Tilde{A}/\sim$
with representatives in $\Tilde{A}$ the strings $\vec{a}=a_{1}\cdots a_{n},
\vec{b}=b_{1}\cdots b_{m}$.
We define $[\vec{a}]^{\frown}[\vec{b}]:=[a_{1}\cdots a_{n}b_{1}\cdots b_{m}]$.
Prove that this is a well-defined operation and that $\Tilde{A}/\sim$ together
with the operation $^\frown$ is a group. We call $\Tilde{A}/\sim$ 
together with the aforementioned operation the
{\em free group} on the set $A$. Prove that the assignment
\begin{equation*}
    A\longmapsto \Tilde{A}/\sim
\end{equation*}
is a part of a functor $\textbf{Sets}\to\textbf{Grp}$. This functor is called the
{\em free functor.}
\end{exercise}

\begin{exercise}\label{exc:prodcat}
Prove that the product category defined in \autoref{examples:functors} is 
indeed a category. Define properly the functors $\pi_{0}:\cC\times\cD\to\cC$
and $\pi_{1}:\cC\times\cD\to\cD$.
\end{exercise}


\begin{exercise}\label{exc:slicecat}
Let $\cC$ be a category and $X$ and object of $\cC$. Prove that the slice category 
$\cC/X$ defined in \autoref{def:slicecat} is indeed a category.
Prove that the assignment defined in \autoref{slice:func} indeed 
defines a functor. Moreover, show that this assignment
\begin{equation*}
    X\mapsto \cC/X
\end{equation*}
gives rise to a functor $\cC\to\textbf{Cat}$.
\end{exercise}


\begin{exercise}\label{exc:oppcat}
Convince yourself that the statements in \autoref{note:opposite}(b) are true.
\end{exercise}

\begin{exercise}\label{exc:yonedafunc}
Prove that the functors $y_{A},h_{A}$ defined in \autoref{example:yoneda} are indeed
functors.
\end{exercise}

\begin{exercise}[$\Hom$ functor] \label{exc:homfunc}
Let $\cC$ be a locally small category. Show that there is a functor 
$\Hom(-,-):\cC^{\op}\times\cC\to\textbf{Set}$, assigning $\Hom(A,B)$
to the pair $(A,B)$ of objects of $\cC$.
\end{exercise}



\begin{exercise}[Abelianization functor] \label{exc:abelianizationfunc}
Let $G$ be a group and consider the {\em commutator subgroup} $[G,G]$ of $G$,
generated by all elements of the form $[g,h]:=ghg^{-1}h^{-1}, g,h\in G$.
Prove that $[G,G]$ is a normal subgroup of $G$, that $G/[G,G]$ is an
abelian group and satisfies the following property: 
for every group homomorphism $\phi:G\to H$ to
an abelian group $H$ there exists a unique homomorphism
$\overline{\phi}:G/[G,G]\to H$ such that the diagram
\begin{equation*}
    \begin{tikzcd}
        G \arrow[r,"p"] \arrow[rd,"\phi"']
        & G/[G,G] \arrow[d,"\overline{\phi}"] \\
        & A
    \end{tikzcd}
\end{equation*}
commutes\footnote{This property is the {\em universal property of the
abelianization}.} (here $p:G\to G/[G,G]$ denotes the quotient group homomorphism,
i.e. the map $G\ni g\mapsto g+[G,G]\in G/[G,G]$). The abelian group
$G/[G,G]$ is called the {\em abelianization} of $G$. Show that this construction
 is functorial, i.e. gives rise to a functor $\textbf{Grp}\to \textbf{Abgp}$.
\end{exercise}







\end{document}